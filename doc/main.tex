\documentclass{article}
\usepackage{xgreek}
\usepackage{xunicode}
\usepackage{xltxtra}
\usepackage{graphicx}
\setmainfont[Mapping=tex-text]{GFS Didot}
\begin{document}
\makeatletter
\def\clap#1{\hbox to 0pt{\hss #1\hss}}%
\def\ligne#1{%
  \hbox to \hsize{%
    \vbox{\centering #1}}}%
\def\haut#1#2#3{%
  \hbox to \hsize{%
    \rlap{\vtop{\raggedright #1}}%
    \hss
    \clap{\vtop{\centering #2}}%
    \hss
    \llap{\vtop{\raggedleft #3}}}}%
\def\bas#1#2#3{%
  \hbox to \hsize{%
    \rlap{\vbox{\raggedright #1}}%
    \hss
    \clap{\vbox{\centering #2}}%
    \hss
    \llap{\vbox{\raggedleft #3}}}}%
\def\maketitle{%
  \thispagestyle{empty}\vbox to \vsize{%
    \haut{}{\@blurb}{}
    \vfill
    \ligne{\Large \@title}
    \vspace{5mm}
    \ligne{\textsl{\@author}}
    \vspace{1cm}
    \vfill
    \vfill
    \bas{}{\@location, \@date}{}
    }%
  \cleardoublepage
  }
\def\date#1{\def\@date{#1}}
\def\author#1{\def\@author{#1}}
\def\title#1{\def\@title{#1}}
\def\location#1{\def\@location{#1}}
\def\blurb#1{\def\@blurb{#1}}
\date{\today}
\author{}
\title{}
\location{}
\blurb{}
\makeatother
  \title{Πρόγραμμα χρωματισμού γράφου με χρήση γενετικών αλγορίθμων}
  \author{Κουζούπης Αντώνης, \textsc{Π/06073}\\\emph{<kouzoupis.ant@gmail.com>}}
  \date{\today}
  \location{Πειραιάς}
  \blurb{%
    Πανεπιστήμιο Πειραιώς \\
    Τμήμα Πληροφορικής \\[1em]
    \begin{center}
    \includegraphics{unipi_logo.jpg} \\[4em]
    \end{center}
    ``Τεχνητή Νοημοσύνη και Έμπειρα Συστήματα''
    }%
\maketitle

\section{Περιγραφή Αλγορίθμου}
Σε αυτή την ενότητα θα περιγράψουμε τη λειτουργία του αλγορίθμου (top-level
design) που χρησιμοποιεί το πρόγραμμα.

Το πρόγραμμα γενικά υπολογίζει οργανισμούς μέσω γενετικών αλγορίθμων. Κάθε
οργανισμός αποτελείται από ένα γονίδιο, πίνακα με 0 στα σημεία που δεν είναι 
χρωματισμένα και 1 στα σημεία που είναι χρωματισμένα. Επίσης κάθε οργανισμός
έχει μία τιμή αποδοτικότητας (fitness) η οποία προσδιορίζει κατά πόσο ``κοντά''
είναι το γονίδιο ενός οργανισμού με αυτό του επιθυμητού -- τελικού. Η τιμή του
κυμαίνεται μεταξύ 0 και 1, με το 0 να είναι η χειρότερη αποδοτικότητα και το 1 η
καλύτερη.

Αρχικά, δημιουργείται ένας πληθυσμός από οργανισμούς με τυχαία γονίδια.
Υπολογίζεται η αποδοτικότητα για κάθε οργανισμό του πληθυσμού. Στη συνέχεια
επιλέγεται ένα ποσοστό του πληθυσμού με την καλύτερη αποδοτικότητα, επιλέγονται
δύο γονείς, ζευγαρώνουν και προκύπτουν άλλοι δύο οργανισμοί. Οι καινούργιοι
οργανισμοί θεωρούνται η καινούργια γενιά.

Από την καινούργια γενιά επιλέγεται ένα ποσοστό το οποίο θα υποστεί κάποια
μετάλλαξη. Η παραπάνω διαδικασία συνεχίζεται έως ότου κάποιος οργανισμός έχει
αποδοτικότητα 1, δηλαδή συμπίπτει απόλυτα με το επιθυμητό αποτέλεσμα.

\section{Περιγραφή Προγράμματος}
Στην ενότητα αυτή θα περιγράψουμε αναλυτικά την υλοποίηση του παραπάνω
αλγόριθμου. Αρχικά να αναφέρουμε ότι το πρόγραμμα είναι υλοποιημένο στη γλώσσα
\emph{C}. Αποτελείται από τρία αρχεία. Στο αρχείο \emph{color\_graph.c} υπάρχει
η υλοποίηση του γενετικού αλγορίθμου, στο αρχείο \emph{linked\_list.c} υπάρχει η
υλοποίηση των συναρτήσεων μονά συνδεδεμένης λίστα που χρησιμοποιείται από τον
παραπάνω αλγόριθμο. Τέλος στο αρχείο \emph{proto.h} ορίζονται κάποιες μεταβλητές
για ευκολία στην απομνημόνευση, ορίζονται κάποιες δομές δεδομένων που
χρησιμοποιούνται, ένας πίνακας με το επιθυμητό -- τελικό αποτέλεσμα καθώς και τα
πρωτότυπα διάφορων συναρτήσεων.

\subsection{proto.h}
Ας ξεκινήσουμε την περιγραφή από το αρχείο \emph{proto.h} καθώς περιέχει
μεταβλητές και δομές. Αρχικά γίνονται define κάποιες σταθερές για ευκολία στην
απομνημόνευση αλλά και στην εύκολη τροποποίησή τους. Αυτές οι μεταβλητές είναι
το μέγεθος του πίνακα των γονιδίων (\emph{COL, ROW}), ο αριθμός του πληθυσμού
(\emph{POPULATION}), το ποσοστό ανανέωσης του πληθυσμού (\emph{POP\_RATE}), η
μάσκα διασταύρωσης (\emph{HSIZE}), το ποσοστό μετάλλαξης του πληθυσμού
(\emph{MUT\_RATE}) κτλ.

Στη συνέχεια ορίζεται η δομή \emph{Organisms} η οποία αντιπροσωπεύει έναν
οργανισμό. Όπως έχουν ήδη αναφέρει, κάθε οργανισμός αποτελείται από ένα γονίδιο
που είναι ένας πίνακας διαστάσεων \emph{ROW}x\emph{COL} με 0 και 1 (\emph{gene}),
και έναν αριθμό που προσδιορίζει την αποδοτικότητα του (\emph{fitness}).

Έπειτα είναι ορισμένη η δομή \emph{Node} η οποία χρησιμοποιείται από τη
συνδεδεμένη λίστα και αντιπροσωπεύει ένα κόμβο. Κάθε κόμβος αποθηκεύει έναν
οργανισμό (\emph{Organisms}) και ένα δείκτη για τον επόμενο κόμβο.

Επόμενο στη σειρά είναι ένας πίνας διαστάσεων \emph{ROW}x\emph{COL} ο οποίος
περιέχει 0 και 1 σε τέτοια διάταξη ώστε να σχηματίζουν τον γράμμα \textbf{K} με
την προϋπόθεση να μην υπάρχουν 1 σε γειτονικά κελιά. Αυτός ο πίνακας
χρησιμοποιείται για τη σύγκριση και τον υπολογισμό της αποδοτικότητας.

Τέλος υπάρχουν τα πρωτότυπα των συναρτήσεων που χρησιμοποιούνται και στα δύο
αρχεία (\emph{color\_graph.c}, \emph{linked\_list.c}).

\subsection{linked\_list.c}
Στο αρχείο αυτό υπάρχει η υλοποίηση μιας μονά συνδεδεμένης λίστα. Για τον λόγο
αυτό δεν θα το περιγράψουμε αναλυτικά καθώς δεν είναι αυτός ο σκοπός του
μαθήματος.

Η συνάρτηση \emph{push()}, παίρνει ως όρισμα το πρώτο στοιχείο μιας λίστας και
ένα αντικείμενο τύπου \emph{Organisms} και το βάζει στο τέλος της λίστας. Η
συνάρτηση \emph{copy\_list()} παίρνει ως όρισμα τα πρώτα στοιχεία δύο λιστών και
αντιγράφει τα στοιχεία της πρώτης στη δεύτερη. Η \emph{delete()} παίρνει ως
όρισμα το πρώτο στοιχείο μιας λίστας και διαγράφει όλα τα στοιχεία της. Η
συνάρτηση \emph{print\_list()} τυπώνει τα στοιχεία μιας λίστας, χρησιμοποιείται
για debugging. Η συνάρτηση \emph{sort()} παίρνει ως όρισμα το πρώτο στοιχείο
μιας λίστας και την ταξινομεί σε φθίνουσα σειρά. Τέλος η συνάρτηση \emph{size()}
παίρνει ως όρισμα τον πρώτο κόμβο μιας λίστα και επιστρέφει το μέγεθός της.
\end{document}
